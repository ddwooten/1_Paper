\documentclass[]{elsarticle}

% bold symbol shortcut
\def\bm#1{\boldsymbol{#1}}

\usepackage{lineno,hyperref}
\modulolinenumbers[5]

%For inclusion of figures
\usepackage{graphicx}

%For isotope notation
\usepackage{mhchem}

%For equation formatting
\usepackage{amsmath}

% math fonts
\usepackage{amsfonts}

%For sizing summations properly
\usepackage{relsize}

%For tables
\usepackage{multirow}

% For nice matrices
\usepackage{blkarray}

%For signalling my co-author
\usepackage{xcolor}

%For landscape tables
\usepackage{pdflscape}

%For footnotes in tables
\usepackage{tablefootnote}

%For appendicies
\usepackage[page]{appendix}

%For fixing floats in place
\usepackage{float}

% for mathfrak?
\usepackage{eufrak}

%for style!
\def\j{\mathfrak{j}}

\journal{Journal of Computational Physics}

%%%%%%%%%%%%%%%%%%%%%%%
%% Elsevier bibliography styles
%%%%%%%%%%%%%%%%%%%%%%%
%% To change the style, put a % in front of the second line of the current style and
%% remove the % from the second line of the style you would like to use.
%%%%%%%%%%%%%%%%%%%%%%%

%% Numbered
%\bibliographystyle{model1-num-names}

%% Numbered without titles
%\bibliographystyle{model1a-num-names}

%% Harvard
%\bibliographystyle{model2-names.bst}\biboptions{authoryear}

%% Vancouver numbered
%\usepackage{numcompress}\bibliographystyle{model3-num-names}

%% Vancouver name/year
%\usepackage{numcompress}\bibliographystyle{model4-names}\biboptions{authoryear}

%% APA style
%\bibliographystyle{model5-names}\biboptions{authoryear}

%% AMA style
%\usepackage{numcompress}\bibliographystyle{model6-num-names}

%% `Elsevier LaTeX' style
\bibliographystyle{elsarticle-num}
%%%%%%%%%%%%%%%%%%%%%%%

\begin{document}

\begin{frontmatter}

\title{Towards Optimization of Material Flows in Molten Salt Nuclear Reactors}


%% Group authors per affiliation:
\author[ucb]{\corref{cor1}Daniel Wooten}
\ead{danieldavidwooten@gmail.com}
\author[ucb]{Massimiliano Fratoni}
\address[ucb]{4155 Etcheverry Hall, MC 1730, University of California, Berkeley,
    Berkeley, CA 94720-1730, United States}
\cortext[cor1]{Corresponding Author}


\begin{abstract}
Molten salt reactors have seen a resurgence of interest in the last 20 years and
yet the options available for fuel evolution modelling in many of these systems
are few with many of them making course assumptions and many more being poorly
documented. ADDER, the Advanced Dynamic Depletion Extension for Reprocessing,
is a source code modification to the SERPENT 2 monte carlo reactor physics code
which seeks to improve on many of the shortcomings found in the field of
nuclear material evolution modelling. Specifically, ADDER provides an interface
with which to define collections of elements, isotopes, and chemicals. ADDER
provides a means to specify how these collections should interact within a
SERPENT 2 material and how these collections may be moved in, out, and between
SERPENT 2 materials. Furthermore, ADDER optimizes optional material flows based
on user defined material constraints and user defined material flows. Two 
specific material constraints that ADDER provides an interface for are
limitations on the multiplication factor of a system in SERPENT 2 as well as
limitations on the averaged and weighted oxidation state of a material in
SERPENT 2. Taking all these considerations into account the optimization
routines of ADDER provides optimized material flows  which are incorporated
directly into the SERPENT 2 burnup routines. ADDER is expected to significantly
advance the current state of molten salt reactor fuel evolution modelling and
possibly other nuclear evolution modelling problems.
\end{abstract}

\begin{keyword}
\texttt{MSR molten salt dynamics burnup optimization}
\MSC[2010] 00-01\sep  99-00
\end{keyword}

\end{frontmatter}

\linenumbers

\section{Introduction} \label{sec:intro}
Molten Salt Reactors (MSRs) have received scattered and intermittent attention
from the global community for the past 70 years. Two MSRs have ever been built,
the Aircraft Reactor Experiment and the Molten Salt Reactor Experiment, both
in the United States though continued research efforts were shuttered in the
1970s. Since the 1990s various pre-conceptual designs have been put forward
by both governments and private industry as, once again, MSRs receive increased
global attention. The Molten Salt Fast Reactor (MSFR) from the European Union,
the Molten Salt Actinide Recycler and Transmuter from Russia, the 
Chinese-designed Liquid Fueled Thorium Molten Salt Reactor, the
FUJI series of reactors out of Japan, and a host of proposed privately designed
reactors are all examples of such pre-conceptual designs.

Despite this depth
of efforts there are few tools available to researchers for nuclear fuel
depletion simulation in molten salt, or even circulating fuel, systems. Those
that do exist suffer various shortcomings - most commonly that few if any
details are provided in literature concerning the methods themselves. 
Many, if not most, of the methods
proposed or implemented involve external linking-scripts such as those seen in
\cite{jeong_development_2015}, \cite{mitachi_three-region_2007},
\cite{nuttin_potential_2005},  \cite{ridley_method_2017}, and
\cite{sheu_depletion_2013}. These scripts 
join together a neutronics solver with a burnup solver. Many of these scripts
tend to ''reprocess" the MSR fuel as well, modifying the material isotopics and 
densities to
reflect possible reactor operations. Linking-scripts, while quite flexible, pay
a price both in computational time and in version complexity.
Other common shortcomings include  
reactor specific methods, such as those seen in \cite{aufiero_extended_2013} and
\cite{ridley_method_2017}, and the use of rigid assumptions about fuel salt
chemistry, such as those found in \cite{fiorina_molten_2013} and 
\cite{ridley_method_2017}. Additionally, across many of the methods available
today 
there is a general lack of attention to the limits and demands of the chemistry
of the fuel salt. Concerns such as the molar fraction of various salt species
in relation to one another or the maintenance of the fuel salt redox potential are
generally treated with hard-limits, upper cut-offs, and coarse assumptions.

Clearly, a need exists within the MSR community for a more nuanced and
system-agnostic fuel evolution suite. In this paper the methodology for such an
approach is presented. Dubbed ADDER, the Advanced Dynamic Depletion
Extension for Reprocessing, it is a suite of tools, agnostic in their
preferred system, built into the monte-carlo reactor physics code SERPENT 2
\cite{leppanen_serpent_nodate}.
ADDER should not be thought of as an extension \textit{for} MSRs, as it's
capabilities open possibilities beyond MSR research. Rather, ADDER is a
``materials optimization" extension for SERPENT 2 with capabilities to address
many of the shortcomings within the current MSR modelling environment. Right off
the bat ADDER benefits from being a source-code modification to the extensive
and widely-used SERPENT 2 code. The SERPENT 2 code already has a long list of
papers attesting to its capabilities as both a neutronics modelling suite and
as a nuclear depletion suite. With ADDER being directly integrated into the
flow of the SERPENT 2 program all the concerns regarding linking-scripts fall
away, but ADDER goes far beyond simply getting rid of a linking-script.

ADDER brings to the users of SERPENT 2 the ability to define groups of elements,
isotopes, and chemicals; the ability to define relationships between these 
groups; and the ability to move these groups into, out-of, and between SERPENT 2
materials. Furthermore, ADDER gives the users of SERPENT 2 the ability to set
$k_{eff}$ targets, prescribe mass transfers within the system, and to set
weighted oxidation state targets for materials. Bringing all of these
capabilities together ADDER employs the COIN-OR linear optimization (CLP) 
package to determine the material flows, as defined by the user, that will best 
satisfy an optimization target set by the user \cite{lougee-heimer_common_2003}.
ADDER does all of this as a source-code
modification to SERPENT 2. With these capabilities ADDER brings to the MSR
community a reactor-agnostic, fuel-salt-agnostic, material flows optimizer.
While ADDER is no general-purpose chemistry code and certainly has no ability
to perform tasks such as chemical phase determination it does bring the ability
to quickly and easily describe the basic chemistry concerns of an MSR as well
as the actions available to remedy these concerns to an extensive reactor
modelling environment. 

In section \ref{sec:structure} the components of ADDER will be introduced and elaborated
upon as well as their specific applicability to MSR modelling. Following, in
section REF\_NEEDED, the algorithmic underpinnings of ADDER will be elaborated
upon followed up with concluding remarks.

\section{Defining the Structure} \label{sec:structure}
ADDER is composed of various structures and these structures bring functionality
to ADDER. In this section these structures and their attributes are explored. In
section \ref{sec:opt} how these structures fit together to fulfill the
job of ADDER is shown.

\subsection{Groups} \label{ssec:groups}
The ``group" is an elementary structure in the ADDER framework. Before
understanding what can be done with a group, what a group is must be understood.
A group in ADDER is composed of a fixed set of elements, with or without
specified isotopics, with fixed abundances relative to the group as a whole.
A group could be made to specify that it is one part uranium and three parts
chlorine - uranium tetrachloride. Furthermore, the uranium could be specified
to be 4.95\% \ce{^{235}U} and 95.05\% \ce{^{238}U} - or left to assume
whatever isotopics the host material may impart to it. 
Finally, related to the group structure in ADDER, is the concept of free versus
``controlled" elements and isotopes. In ADDER, for a given material, whether or
not an element or isotope should be completely accounted for by the groups
which possess these constituents or be allowed to have ``free" portions
not locked up in the group structure is something that can be specified.

With a set of groups assigned to a material in SERPENT 2 there are several means
through which to define the relationships between the groups and between the
material and the groups. To define relationships
between the material and the groups a range of absolute abundances of a group
in a material may be specified for an arbitrary number of groups. 
To define relationships between groups in a SERPENT 2
material a range of relative abundances between any two groups
may be defined for an arbitrary number of group pairs. To facilitate modelling
of chemical compounds with related and possibly interchangeable forms a group
in ADDER may also be formed from the linear combination of any other groups
previously defined. These three simple mechanisms can be combined to model
various chemical situations and form a core component of the conditions for
optimality placed on a material. 

Say, for example, that a material is desired to have three to four times as
much eutectic LiF - Be\ce{F2} (FLIBE) salt to uranium fluoride salts - both 
uranium trifluoride and tetrafluoride. Additionally, it is desired that uranium
tetrafluoride be more than 100 times as abundant as uranium trifluroide. Setting
these as constraints for the material can be accomplished with four groups and
2 relative abundance constraints. First the uranium trifluoride and tetrafluoride
groups are defined followed by a one to one combination of the two uranium
groups to form an overall ``uranium fluoride compounds" group. Of course a FLIBE
salt group is made as well. A relative abundance constraint is placed between
the FLIBE group and the ``uranium fluoride compounds" group and a relative
constraint is placed between the two uranium fluoride salt groups. With those
6 constructs a solubility constraint on a family of related compounds was put
into effect. This is just once example of many restraints and conditions which
can be modeled with the ADDER group structures and the relationships between
them.

\subsection{Oxidation tables} \label{ssec:oxi}
In many nuclear fuel evolution scenarios a wide variety of elements are
produced by nuclear processes. To ease the burden on the user for
defining how all of these elements will bond with one another the ``oxidation
table" structure is provided. In the oxidation table structure a complete list
of elements with their expected average oxidation state in the desired
material is given; optionally a weight may also be applied to any element. To complete this setup a material is also given a target
oxidation state which the weighted average of the elemental oxidation states in
the material should meet.
It is suspected that this feature will greatly ease the
burden on most MSR simulations as most proposed fuel salts have a dominant
anion which bonds with near everything else to the exclusion of other
compounds. Combining this tool with the principles from the Nernst equation,
equation \ref{eq:nernst}, bounds for the average redox potential of a molten
salt can be set - a key metric in controlling corrosion in molten salt systems.
In equation \ref{eq:nernst} $\epsilon_{i}$ is the redox potential,
$\epsilon_{i}^{o}$ is the redox potential in the standard state, R is the gas
constant, T is the temperature, z is the number of electrons received by the
oxidizing agent, $\j$ is Faraday's constant, $[oxid]$ is the activity of the oxidized species while $[red]$ is the activity of the reduced species.
Conversions between
activity and concentration are required but approximations may be sufficient -
this task is considered far beyond the scope of ADDER and is left to the user.

\begin{equation}
\label{eq:nernst}
    \epsilon_{i} = \epsilon_{i}^{o} + \frac{RT}{z \j}\ln\left(\frac{[oxid]}{[red]}\right)
\end{equation}

\subsection{Streams} \label{ssec:streams}
The ``stream" is both the workhorse and the end-goal of ADDER.
There are two classes of streams in ADDER. ``Group-class" streams are
options. They represent pathways available to ADDER to move mass into, out of,
and between SERPENT materials with the goal of bringing their compositions to an
optimal state. ``Table-class" streams are prescriptive. They are directions to
ADDER to move specific types and amounts of mass from and to specific materials
 - irrespective of optimallity. All streams have a set of common attributes
provided by the user; a source, a sink, and the behavior in time of the stream.
For the majority of streams a source and a sink are optional, but one must be
provided. Missing sources are treated as infinite supplies of whatever substance
is needed, missing sinks are treated much like sinks - endless consumers of
disposed mass. In terms of their behavior in time, ADDER
supports three types of streams. Discrete type stream transfers happen
between burnup steps as step changes. Continuous type stream transfers occur as
a steady rate of mass transfer over the length of a burnup step. Proportional
type streams modify the decay constant
of isotopes, even to the point of making the decay constant a production
constant if that is what is called for. Lastly, the user is given the option in
ADDER to preserve the number of mols in a material - this option can be turned
on or off by the user.

Group-class streams have an additional attribute the user is required to set, 
the ADDER group which defines the substance the stream will move. These streams
are given no set amount of mass transfer. Rather, through its optimization 
process
ADDER determines the amount of mass transfer each group-class stream should
have. Table-class streams have two additional attributes the user is required to
set, the ADDER ``transfer" table to be used and a positive value, denoted
$c^{s}$, the purpose 
of which will be made clear in the following lines.
Transfer tables in ADDER are user defined lists of selected elements
and isotopes all of which have some value attached to them, denoted $c_{k}^{t}$.Multiplying the
value from the transfer table with the value given in the table-class stream
definition gives the fraction of the whole for an individual isotope or
element that will be moved by the table-class stream over the burnup step for
continuous and discrete type streams. For proportional type streams the value
produced by this multiplication will be added to the decay constant of the
appropriate isotopes, or subtracted given its sign. The value of splitting
the table-class stream mass transfer rates into two numbers lies with MSR
modelling. In many proposed MSR designs there is some fuel treatment procedure
which is applied to some fraction of the fuel salt, represented by the value
given in the table-class stream definition. This treatment procedure removes
specific elements with differing effectiveness, as represented by the value
given to each element and isotope contained in a transfer table.

Streams, group-class or table-class, have clear applicability to MSR modelling.
While ADDER's optimization routines, in this instance, should be thought of as
the ``reactor
operator" the streams represent those mass flows in and out of a reactor that
the operator may plan such as an addition of lithium fluoride for maintaining
a desired salt condition or the addition of \ce{^{233}U} for criticality
control. Table-class streams provide a means not only to model possible
fuel salt reprocessing options but also natural process which change the
composition of fuel salts such as the escape of noble gas fission products. 
Outside of MSR modelling streams find other applications ranging from geological
repository modelling to fuel fabrication applications.

\subsection{Reactivity control} \label{ssec:reactivity}
Not all, but most, nuclear simulations are concerned with the multiplication
factor of the system in question. A description of the restrictions and
conditions placed
upon a material in SERPENT 2 would not be complete without a mention to
reactivity constraints. As such users may set system wide reactivity
constraints through ADDER. In order to approximate the effects the streams will
have on reactivity ADDER collects spectrum averaged neutron production cross
sections, $\nu \Sigma_{f}$, for all isotopes in the materials ADDER is
connected to. Leakage values, $p_{l}$, and absorption cross sections,
$\Sigma_{a}$, are provided by SERPENT 2 and from equation \ref{eq:reac}
$k_{eff}$ is approximated - summations over i are over all isotopes.

\begin{equation}
\label{eq:reac}
k_{eff} = (1 - p_{l}) \frac{\sum\limits^{I} \nu_{i} \Sigma_{f}^{i}}{\sum\limits^{I} \Sigma_{a}^{i}}
\end{equation}

While equation \ref{eq:reac} is exact, $k_{eff}$ is only ever approximated;
not just from the error inherent in monte-carlo simulations but from
ignoring that every term in equation \ref{eq:reac} is non-linearlly dependant
on composition. As such it is expected that the reactivity control feature of
ADDER will only behave well for small changes in composition that do not have a
large effect on the neutron flux. Finally, more of a limitation than an
assumption, ADDER has no means to measure the effect on reactivity resulting
from changes in one material interacting neutronincally with another. As such 
ADDER's reactivity control feature only works in situations where one 
material is the dominant driver of criticality in a system. While this could be
limiting in many circumstances MSR modelling is not expected to be one of them.
To address these shortcomings ADDER offer the user the option of setting the
number of max iterations allowed per burnup step to determine the effects of
ADDER's streams. At the end of each iteration a monte-carlo cross section
calculation is repeated to assess the effects of ADDER's actions.

\subsection{Material clusters}
An overarching structure that serves as the primary means of organization in
ADDER is a material cluster. Material clusters are created in ADDER when ADDER
streams link materials together through being sources and sinks. ADDER
automatically identifies material clusters that are created by the user's
stream definitions. Material clusters are necessary if for no other reason than
the equations that describe their nuclear evolution become coupled. 

\section{Optimizing the Outcome} \label{sec:opt}
Before considering an optimization method one should consider the constraints
and governing equations. 

\subsection{Group equations} \label{ssec:group_eq}
Beginning with groups there are 2 principle relationships. The first is the
atom balance equation describing a group's composition as seen in equation
\ref{eq:group_def_ele} for the elemental balance and \ref{eq:group_def_iso} for
the isotopic balance where $g_{i}$ is the fractional abundance of the 
group in the host material, $f_{x}^{i}$ is the fractional abundance of 
component $x$ in group $i$, $E_{j}^{f}$ is the fractional abundance of element
$j$ in the same material as group $i$, and $I_{k}^{f}$ is the fractional 
abundance of isotope $k$ in the same material as group $i$.

\begin{equation}
\label{eq:group_def_ele}
g_{i} = \sum \limits_{j}^{J} f_{E_{j}^{f}}^{i} E_{j}^{f}
\end{equation} 

\begin{equation}
\label{eq:group_def_iso}
g_{i} = \sum \limits_{k}^{K} f_{I_{k}^{f}}^{i} I_{k}^{f}
\end{equation}

Considering the limits on the group's abundance within a material equations
\ref{eq:group_def_ele} and \ref{eq:group_def_iso} become the inequalities seen
in equations \ref{eq:group_def_ele_lim} and \ref{eq:group_def_iso_lim} where
$b_{m}$ and $b_{M}$ are the lower and upper bound respectively.

\begin{equation}
\label{eq:group_def_ele_lim}
b_m \leq \sum \limits_{j}^{J} f_{E_{j}^{f}}^{i} E_{j}^{f} \leq b_{M}
\end{equation} 

\begin{equation}
\label{eq:group_def_iso_lim}
b_{m} \leq \sum \limits_{k}^{K} f_{I_{k}^{f}}^{i} I_{k}^{f} \leq b_{M} 
\end{equation}

The second principle relationship is the relative abundance limits between pairs
of groups.
Relative group abundance limits can be expressed as seen in equation
\ref{eq:rto_ratio} where $r_{m/M}$ indicates a relative abundance minimum and
maximum bound respectively.

\begin{equation}
\label{eq:rto_ratio}
r_{m} \leq \frac{g_{1}}{g_{2}} \leq r_{M} 
\end{equation}

Equation \ref{eq:rto_ratio} can be expressed linearly as two inequalities as
seen in equations \ref{eq:rto_min} and \ref{eq:rto_max}.

\begin{equation}
\label{eq:rto_min}
-\infty \leq -g_{1} + r_{m}g_{2} \leq 0
\end{equation}

\begin{equation}
\label{eq:rto_max}
0 \leq -g_{1} + r_{M}g_{2} \leq \infty
\end{equation}

\subsection{Stream equations} \label{ssec:stream_eq}
Taking $s_{l}$ to be stream $l$, $f_{x}^{l}$ to be the fractional abundance
of component $x$ in stream $l$, $E_{j}^{d}$ to be the absolute change in the 
relative abundance of element $j$ in the host material, and $I_{k}^{d}$ to be 
the absolute change in the relative abundance of isotope $k$ in the host
material the solution for the stream's relative abundance is seen in equations
\ref{eq:stream_eq_ele} and \ref{eq:stream_eq_iso}.

\begin{equation}
\label{eq:stream_eq_ele}
s_{l} = \sum \limits_{j}^{J} f_{E_{j}}^{d} E_{j}^{d}
\end{equation}

\begin{equation}
\label{eq:stream_eq_iso}
s_{l} = \sum \limits_{k}^{K} f_{I_{k}}^{d} I_{k}^{d}
\end{equation}

\subsection{Oxidation equations} \label{ssec:oxid_eq}
The governing inequality for the oxidation table is seen in equation
\ref{eq:oxi} where $O_{m/M}$ is the minimum and maximum averaged oxidation
state bound for a material, $o_{E_{j}}$ is the average oxidation state of
element $j$ in the host material, and $w_{E_{j}}$ is the optional weighting
factor that can be applied to any element. 

\begin{equation}
\label{eq:oxi}
O_{m} \leq \sum \limits_{j}^{J} w_{E_{j}} o_{E_{j}} E_{j} \leq O_{M}
\end{equation}

\subsection{Reactivity equations}
Given lower and upper bounds for the multiplication factor of the system,
$k_{eff}^{min}$ and $k_{eff}^{max}$ respectively equation \ref{eq:reac} can be 
made linear as seen in equations \ref{eq:k_min} and \ref{eq:k_max}.

\begin{equation}
\label{eq:k_min}
0 \geq \frac{k_{eff}^{min}}{(1 - p_{l})} \sum \limits_{k}^{K} \sigma_{a}^{k} I_{k} - \sum \limits_{k}^{K} \nu^{k} \sigma_{f}^{k} I_{k}
\end{equation}

\begin{equation}
\label{eq:k_max}
0 \leq \frac{k_{eff}^{max}}{(1 - p_{l})} \sum \limits_{k}^{K} \sigma_{a}^{k} I_{k} - \sum \limits_{k}^{K} \nu^{k} \sigma_{f}^{k} I_{k}
\end{equation}

\subsection{How to optimize}
Equations \ref{eq:group_def_ele_lim} through \ref{eq:k_max} demonstrate that
the governing equations of optimization for ADDER are linear. Given a linear
set of equations, and a set which in various configurations could be
under-constrained, over-constrained, or equal, the best optimization route is
clear, linear optimization or linear programming as it is sometimes known.
The benefits, computational efficiency and a one to one mapping of solutions
to initial conditions, are desired traits in ADDER. The use of linear 
optimization 
does ask two questions; what is the optimization target and what linear
optimization routine should be used. To answer the former ADDER allows the user
to set the optimization direction, minimization or maximization, as well as
the optimization target from which the user can select such options as; a
specific group in a specific material, a specific stream, all material transfers,
and other options. To solve the linear programming problem ADDER employs the
CLP library \cite{lougee-heimer_common_2003}.

\subsection{Building the matrix}
The CLP library expects a linear programming matrix from ADDER - one built from
all the constituent equations in the ADDER scheme. Before the equations 
presented earlier in this section can be incorporated into the matrix a note
about the effects of table-class streams should be made. As mentioned in section
\ref{ssec:streams} table-class streams are prescriptions; the mass flows they
stipulate are going to happen. This information makes it into the linear
programming matrix in the form of adjustments to the bounds of equations
and variables. These adjustments are denoted
as $r_{x}$ where $r$ is the net positive increase in the fractional abundance of
component $x$ as caused by all table-class streams. It should be noted that
adjustments calculated for proportional removal table-class streams are only
approximations, and sometimes poor approximations, of the actual amount of an
isotope or element that will be removed as nuclear processes change the 
abundance of isotopes in a way that ADDER is currently unaware of.

Figure \ref{fig:opt_matrix}
depicts the scheme for constructing the linear programming matrix. Column bounds
are presented above the appropriate column while row bounds are presented to the
left of the appropriate row. Below the column bounds are the variables which
the columns represent and to the right of the row bounds are the equation
number, if any, of the equation that row represents. For the sake of brevity
the matrix in figure \ref{fig:opt_matrix} is for one material only, a one 
material cluster. The only variables shared between materials are the
group-class streams and the stream equations they are a part of are the only
coupling equations; aside from transfers by table-class streams but those
are only represented in the linear programming matrix, they are handled by
other routines all together. If a second material were to be included in this
matrix then, perhaps, the stream entries in the third and fourth columns would
have non-zero coefficients for some $E_{j}^{d}$ and $I_{k}^{d}$ rows.

Working down the matrix row by row the first row encountered represents equation
\ref{eq:rto_min} with arbitrary groups $g_{1}$ and $g_{2}$ while the next row 
down represents equation \ref{eq:rto_max}. The third row, what will be referred
to as an ``elemental future", row represents the atom balance for element $j$.
The novel column involved here is an elemental future column whose inclusion
in the same row closes the equation. The lower bound on this column, 
$e_{m}^{j}$, represents $0 - r_{E_{j}^{f}}$.
The row bounds $\alpha$ and $\beta$ represent, respectively,
$-\infty$ and $0 + r_{E_{j}^{f}}$. These are the bounds for an
``uncontrolled" element or those elements which are permitted to have portions
of the element not tied up in declared group structures. 
The upper bound is met in the trivial case,
where all terms are zero, and in the case of material injection containing
element $j$. In this instance the group contributions to the equation may be
larger than $E_{j}^{f}$ which would push the result of the equation to be
greater than 0 indicating a lack of $E_{j}^{f}$ - except that the missing amount
of $E_{j}^{f}$ will be brought in by table-class streams that the linear
programming matrix is, otherwise, unaware of. For ``controlled" elements
$\alpha$ and $\beta$ are both equal to $0 + r_{E_{j}^{f}}$, the abundance of 
element $k$ must be accounted for entirely by its presence in group structures.
The fourth row, an ``elemental
delta" row, represents the change in the fractional abundance of element $j$
as caused by all group-class streams. Of course the elemental delta column 
is involved to close the balance. The fifth row, or ``balance" row, is what ties
together $E_{j}^{f}$ and $E_{j}^{d}$. The bounds, $\gamma$ and $\delta$,
are equal and represent $E_{j}^{c} + r_{E_{j}^{f}}$ constituting an atom
balance \textit{in time} where $E_{j}^{c}$ represents the present fractional
abundance of element $j$. The fifth row requires, straightforwardly, that the
``future" amount of an element be equal to the current amount plus any ``delta",
or change, in the element's abundance. The sixth row is an isotopic balance row
requiring that the abundance of an element be equal to the abundance of its
constituent isotopes. $\epsilon$ is defined in equation \ref{eq:epsilon_def}
where $I_{k,E}$ represents an isotope $k$ which is also a member of 
element $E$. 
The following three rows, the seventh, eighth, and ninth,
are the isotopic versions of the elemental future, delta, and balance rows.
$\zeta$ represents
$0 + r_{I_{k}^{f}}$; $i_{M}^{k}$ represents
$0 - r_{I_{k}^{f}}$ ; while $\eta$ and $\theta$ are
equal and represent $I_{k}^{c} + r_{I_{k}^{c}}$ where $I_{k}^{c}$ represents
the present fractional abundance of isotope $k$. In the tenth and eleventh rows
equations \ref{eq:k_min} and \ref{eq:k_max} find representation with $\iota$
and $\kappa$ respectively representing terms of the expanded sum found in the
referenced equations; $\frac{k_{eff}^{min}}{(1-p_{l})} \sigma_{a}^{k} - \nu^{k}
\sigma_{f}^{k}$ and
$\frac{k_{eff}^{max}}{(1-p_{l})} \sigma_{a}^{k} - \nu^{k}
\sigma_{f}^{k}$. Table-class stream effects on reactivity are captured in
$\Lambda$ and $\mu$ as seen in equations \ref{eq:Lambda_bound} and
\ref{eq:mu_bound} respectively. In the twelfth row $\xi$ and $\tau$ capture
the effects of table-class streams as seen in equations \ref{eq:xi_bound} and
\ref{eq:tau_bound} respectively. The thirteenth row, or ``pres" row, exists
when the user instructs ADDER to conserve the number of mols in a material. The
pres row requires that the net stream transfers in a material come to zero. The
effects of table-class streams are captured in $\upsilon$ and $\omega$ as seen
in equation \ref{eq:upsilon_bound} where $s^{t}$ is a table class stream 
abundance value. The final row is the optimization, or ``Opt", row. This row
indicates to the simplex routine which variables to minimize or maximize. In
figure \ref{fig:opt_matrix} the opt row is indicating that $g_{1}$ is the
optimization target. ADDER gives many options to the user for optimization
targets; specific groups, specific streams, sets of streams, specific elements,
specific isotopes, and many others.

    \begin{equation}
    \label{fig:opt_matrix}
        \centering
        \begin{blockarray}{cccccccccc}
                               &                   & [b_{m},b_{M}]     &
            [0, \infty)        & (-\infty,\infty)  & (-\infty,\infty)  &
            [e_{m}^{j}, \infty)& (-\infty,\infty)  & [i_{m}^{j},\infty)&
            (-\infty,\infty)  \\ 
                               &                   & g_{1}             &
            g_{2}              & s_{1}             & s_{2}             &
            E_{j}^{f}          & E_{j}^{d}         & I_{k}^{f}         &
            I_{k}^{d} \\
                               &                   &                   &
                               &                   &                   &
                               &                   &                   &
             \\ 
            \begin{block}{cc[cccccccc]}
            {(-\infty,0]}      & \text{Eq.}\ref{eq:rto_min} & -1       &
            r_{m}              &                   &                   &
                               &                   &                   &
             \\
            {[0,\infty)}       & \text{Eq.}\ref{eq:rto_max} & -1       &
            r_{M}              &                   &                   &
                               &                   &                   &
             \\
            {[\alpha,\beta]}   & E_{j}^{f}   & f_{E_{j}^{f}}^{1} &
            f_{E_{j}^{f}}^{2}  &                   &                   &
            -1                 &                   &                   &
             \\
            {[0,0]}            & E_{j}^{d}         &                   &
                               & f_{E_{j}^{f}}^{1} & f_{E_{j}^{f}}^{2} &
                               & -1                &                   &
             \\
            {[\gamma, \delta]} 
                               & E_{j}^{b}         &                   &
                               &                   &                   &
            1                  & -1                &                   &
             \\
            {[\epsilon,\epsilon]}& E_{j}^{i}       &                   &
                               &                   &                   &
            -1                 &                   & 1                 &
             \\
            {(-\infty,\zeta]} & I_{k}^{f}         & f_{I_{k}^{f}}^{1} &
            f_{I_{k}^{f}}^{2}  &                   &                   &
                               &                   & -1                &
             \\
            {[0,0]}            & I_{k}^{d}         &                   &
                               & f_{I_{k}^{f}}^{1} & f_{I_{k}^{f}}^{2} &
                               &                   &                   &
            -1 \\
            {[\eta, \theta]}
                               & I_{k}^{b}         &                   &
                               &                   &                   &
                               &                   & 1                 &
            -1 \\ 
            {[\Lambda, \infty)} & \text{Eq.}\ref{eq:k_max}&            &
                               &                   &                   &
                               &                   & \iota             &
             \\
            {(-\infty, \mu]}   & \text{Eq.}\ref{eq:k_min} &            &
                               &                   &                   &
                               &                   & \kappa            & 
             \\
            {[\xi,\tau]}    & \text{Eq.}\ref{eq:oxi} &                 &
                               &                   &                   &
             o_{E_{j}^{f}}     &                   &                   &
             \\
            {[\upsilon,\omega]} & \text{Pres}      &                   &
                               & 1                 & 1                 &
                               &                   &                   &
             \\
                               & \text{Opt}        & 1                 &
                               &                   &                   &
                               &                   &                   &
             \\
            \end{block}
        \end{blockarray}
    \end{equation}

\begin{equation}
\label{eq:epsilon_def}
\epsilon = \sum \limits_{k}^{K} I_{k,E} r_{I_{k,E}^{f}}
\end{equation}

\begin{equation}
\label{eq:Lambda_bound}
\Lambda = \sum \limits_{k}^{K} \nu^{k} \sigma_{f}^{k} r_{I_{k}^{f}} -
    \frac{k_{eff}^{min}}{(1 - p_{l})} \sum \limits_{k}^{K} \sigma_{a}^{k}
    r_{I_{k}^{f}}
\end{equation}

\begin{equation}
\label{eq:mu_bound}
\mu = \sum \limits_{k}^{K} \nu^{k} \sigma_{f}^{k} r_{I_{k}^{f}} -
    \frac{k_{eff}^{max}}{(1 - p_{l})} \sum \limits_{k}^{K} \sigma_{a}^{k}
    r_{I_{k}^{f}}
\end{equation}

\begin{equation}
\label{eq:xi_bound}
\xi = O_{m} - \sum \limits_{j}^{J} w_{E_{j}} o_{E_{j}} r_{E_{j}^{f}}
\end{equation}

\begin{equation}
\label{eq:tau_bound}
\tau = O_{M} - \sum \limits_{j}^{J} w_{E_{j}} o_{E_{j}} r_{E_{j}^{f}}
\end{equation}

\begin{equation}
\label{eq:upsilon_bound}
\upsilon = \omega = -\sum \limits_{s'}^{S'} s_{s'}^{t}
\end{equation}

\subsection{Solving the matrix} \label{ssec:solving}
ADDER builds the matrix seen in figure \ref{fig:opt_matrix} and passes this
matrix to CLP. CLP solves the linear programming problem and returns back a
vector containing the value of the objective function as well as the values
all the variables take in the optimal solution. The critical information from
this process are the values of the stream abundances.

\section{Burning the Material}
Following the solution of the optimization problem discrete type streams have
their effects applied before the burnup step begins. A monte-carlo simulation is
then run and if the multiplication factor is outside of the user defined bounds
and iterations remain, as set by the user, another optimization solve will be
executed except the changes already made by discrete type streams remain. 
Beyond this the effects that ADDER has are to modify the burnup matrix inside of
SERPENT 2 to reflect the effects of streams. The coefficients in the burnup
matrix are those in the Bateman equation as seen in equation \ref{eq:Bateman}
where $N$ is the number density of nuclide $n$, $\vec{r}$ is the position at
which the nuclide density is taken, $t$ is time, $b_{m \to n}$ is the branching ratio for the decay of nuclide $m$ into $n$, $\lambda$ is the decay constant
for its sub-scripted nuclide,
$q$ goes over all neutron induced absorption reactions for a given isotope, 
$a_{m \to n}^{q}(E)$ is the branching ratio for isotope $m$ into $n$ due to
reaction $q$ at energy $E$,  $\sigma_{x}^{y}(E)$ is the microscopic
cross section of reaction $x$ for isotope $y$ at energy $E$, $\phi$ is the
scalar neutron flux, $d$ denotes all
transmutation reactions for a given isotope, $R_{n}(\vec{r},t)$ is a
fractional removal (or addition) rate for isotope $n$ at position $\vec{r}$ at
time $t$, and $F_{n}(\vec{r},t)$ is a feed (or removal) amount for isotope
$n$ at position $\vec{r}$ at time $t$. 

A highly truncated burnup scheme can be
seen in figure \ref{fig:burn_matrix} in which there are two isotopes,
\ce{^{233}U} and \ce{^{135}Xe}, and two streams; $S_{c}$ representing a
``continuous" stream with a constant injection rate and $S_{p}$ representing a
``proportional" stream with a transfer rate dependent upon the concentration
of the substances to be transferred. There are, of course, two matrices as well.
The burnup matrix to the left holding the coefficients of the Bateman equation
and the second, to the right, holding the initial concentrations of isotopes
and the values for the streams. The first column of the first row gives
the creation and destruction of \ce{^{233}U} which is dependant on the
concentration of \ce{^{233}U} with $\Gamma$ representing nuclear destruction as
seen in equation \ref{eq:Gamma_def}. The third column of the first row holds
the fraction of stream $S_{c}$ that \ce{^{233}U} comprises. These entries
together describe the evolution of \ce{^{233}U} in the given system. In the
second row $\Xi$, as seen in equation \ref{eq:Xi_def}, represents the production
of \ce{^{135}Xe} from \ce{^{233}U}. In the second column of the second row are
the processes dependant on the concentration of \ce{^{135}Xe}. $\Upsilon$
represents the proportional rate constant as determined by the multiplication
of $c_{\ce{^{135}Xe}}^{t}$ and $c^{s}$ while $\Theta$ is given by equation
\ref{eq:Theta_def}. The third row is blank as the abundance of a continuous type
stream, $h_{S_{c}}$, does not change over a burn step. The fourth row is an
addition specific to ADDER and not found in the Bateman equations; rather, this
line, and the lines it represents, exists to keep track of the amount of an
isotope that a proportional stream moves simply to have this information to give
to the user. The system of matrices seen in figure \ref{fig:burn_matrix} is
solved by SERPENT providing updated isotopic abundances and proportional stream
transfer amounts.

    \begin{equation}
    \label{eq:Bateman}
    \begin{split}
        \frac{\partial N_{n}(\vec{r},t)}{\partial t} = & \sum \limits_{m}^{M} 
        b_{m \rightarrow n} \lambda_{j} N_{j}(\vec{r}, t) + \\
        & \sum \limits_{m}^{M}
        \sum \limits_{q}^{Q} \int_{0}^{\infty} a_{k \rightarrow i}^{q}(E)
        \sigma_{q}^{k}(E) \phi(\vec{r},E,t) N_{k}(\vec{r},t)\mathrm{d}E - \\
        & N_{n}(\vec{r},t) \lambda_{i} - \sum \limits_{d}^{D} \int_{0}^{\infty}
        \sigma_{d}^{n}(E) \phi(\vec{r},E,t) N_{n}(\vec{r},t)\mathrm{d}E - \\
        & R_{n}(\vec{r},t) N_{n}(\vec{r},t) + F_{n}(\vec{r},t)
    \end{split}
    \end{equation}

    \begin{equation}
    \label{fig:burn_matrix}
        \begin{blockarray}{cccccc}
             &
            \ce{^{233}U} &
            \ce{^{135}Xe} &
            S_{c} &
            S_{p} &
            \mathbb{N} \\
             &
             &
             &
             &
             &
             \\ 
        \begin{block}{c[cccc][c]}
            \ce{^{233}U} &
            -\lambda_{\ce{^{233}U}} + \Gamma &
             &
            f^{S_{c}}_{\ce{^{233}U}} &
             &
            N_{\ce{^{233}U}} \\
            \ce{^{135}Xe} &
            \Xi &
            -\lambda_{\ce{^{135}Xe}} + \Upsilon + \Theta &
             &
             &
            N_{\ce{^{135}Xe}} \\
            S_{c} &
             &
             &
             &
             &
            h_{S_{c}} \\
            S_{p} &
             &
             \Upsilon &
             &
             &
             0\\
        \end{block}
        \end{blockarray}
    \end{equation}

\begin{equation}
\label{eq:Gamma_def}
\Gamma = - \sum \limits_{d}^{D} \sigma_{d}^{\ce{^{233}U}} \phi
\end{equation}

\begin{equation}
\label{eq:Xi_def}
\Xi = b_{\ce{^{233}U} \rightarrow \ce{^{135}Xe}} \lambda_{\ce{^{233}U}} + \sum
\limits_{q}^{Q} a_{\ce{^{233}U} \rightarrow \ce{^{135}Xe}} \sigma_{q}^{
\ce{^{233}U}} \phi
\end{equation}

\begin{equation}
\label{eq:Theta_def}
\Theta = - \sum \limits_{d}^{D} \sigma_{d}^{\ce{^{135}Xe}} \phi
\end{equation}

\section{Iterating to the Solution} \label{sec:iter}
The Bateman equation, \ref{eq:Bateman}, is not a standalone description of the
isotopic evolution of a nuclear system; rather, it is tightly coupled with the
scalar neutron flux. Considering that the solution of the system of matrices
in figure \ref{fig:burn_matrix} does not solve for the scalar neutron flux
it is clear that the solution, if for nothing else, is an approximation. Many
nuclear material evolution schemes iterate between solutions of the neutron
flux and the Bateman equations within the same burnup step - SERPENT 2 is no 
different. While the purpose of
this paper is not to investigate the burnup solution routines of SERPENT, those
can be seen in \cite{leppanen_burnup_2009}, the iteration scheme employed by
these routines could affect ADDER. In truth, ADDER is compatible with any
iteration scheme employed by SERPENT 2 due in part to a limitation of ADDER -
due to the mix of continuous and discrete streams convergence of an iteration
scheme for optimization involving nuclear processes on the fuel is impossible 
to guarantee. At
the present time ADDER only iterates to check the reactivity component of its
solution. These iterations happen at the beginning of every burn step in which
the Bateman equations will be solved. Other than these actions, ADDER does not
interact with SERPENT 2 burnup iterations schemes.

\section{Conclusion} \label{sec:conc}
The current state of nuclear materials evolution methods is lacking; both in
the ability to handle material flows and in the ability to determine the correct
material flows. ADDER has been presented as a step in the right direction
towards addressing these concerns. ADDER, a source code modification to
SERPENT 2, does improve on the current state
in the diversity of the material transfers it allows, in the diversity of
the material transferred that it allows, and in the optimization of these
material flows according to a wide array of constraints; yet, like all methods,
ADDER has its limitations. ADDER is not a full chemistry simulation suite, it
barely scratches the surface of the complex chemical relationships that exist 
in many nuclear systems. Furthermore, ADDER effectively ignores the coupling 
between the nuclear evolution problem and the optimization problem. Despite
these shortcomings ADDER is expected to contribute significantly to MSR fuel
evolution modelling and possibly to a lesser extent other nuclear modelling
situations involving material flows such as geologic repository or
fuel fabrication facility analysis. 

\section*{References}

\bibliography{theory}

\end{document}
